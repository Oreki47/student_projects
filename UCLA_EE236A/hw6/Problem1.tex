\subsection*{Problem 1}
\subsubsection*{(1, 2)}
\paragraph{}
See attached sheet for Steiner tree graphs.
\subsubsection*{(3)}
\paragraph{}
Let $T=\{T_1, T_2,...,T_m\}$ be the Steiner trees found in graph $G$. Each tree $T_i$ has a flow $f_i$. Then an LP could be formed as followed:
\begin{align*}
& maximum \quad \ \ \sum_{i \in T} f_i\\
& subject \ to \qquad f_i \geq 0, \ i = 1,...,m \\
&\qquad \qquad \quad \quad \sum_{i:e\in T_i} f_i \leq C_e, \ \forall e \in E
\end{align*}
\subsubsection*{(4)}
\paragraph{}
The corresponding dual of this problem could be formed as followed:
\begin{align*}
& minimize \quad \ \ \  \sum_{e\in E} c_e d_e\\
& subject \ to \qquad d_e \geq 0, \ \forall e \in E  \\
&\qquad \qquad \quad \quad \sum_{e:e\in T_i} d_e \geq 1, \  i = 1,...,m
\end{align*}
\paragraph{}
This is similar to max-flow, min-cut problem with variable being the path except here the variable is the Steiner tree. Therefore the dual variable $d_e$ can be seen as the distance between $s$ and $r$ with its minimum value being 1. This is to say that any cut given by the dual will seperate $s$ and $r$.