\subsection*{Section 3}
\paragraph{}
Table \ref{table:1} list all nodes with degree larger than 200, with an average $m_{core\_nodes} = 279.375$. There are a total of 40 core nodes.
\begin{table}[h!]
	\centering
	\caption{List of Core Nodes}
	\begin{tabular}{{c}*{11}{c}}
		\hline
		& 1    & 108  & 349  & 484  & 1087 & 1200 & 1353 & 1432 & 1585 & 1590  \\ 
		\hline
		& 1644 & 1685 & 1731 & 1747 & 1769 & 1801 & 1828 & 1889 & 1913 & 1942  \\
		\hline
		& 1986 & 1994 & 2048 & 2079 & 2124 & 2143 & 2207 & 2219 & 2230 & 2234  \\
		\hline
		& 2241 & 2267 & 2348 & 2411 & 2465 & 2508 & 2544 & 2561 & 2612 & 3438  \\
		\hline
	\end{tabular}
	\label{table:1}
\end{table}
\paragraph{}
Since Node 1 is also a core node, we will be studying this node for problem 3. Figure \ref{fig:d} to \ref{fig:f} presents the community structure by \textit{Fast Greedy, Edge-Betweenness,} and \textit{Infomap} respectively. 

\begin{figure}[h!]
	\centering
	\includegraphics[width=.6\linewidth]{q1d.png}
	\caption{Node 1 Community Structure by \textit{fast greedy algorithms}}	
	\label{fig:d}
\end{figure}
\begin{figure}[h!]
	\centering
	\includegraphics[width=.6\linewidth]{q1e.png}
	\caption{Node 1 Community Structure by \textit{Edge-Betweenness algorithms}}	
	\label{fig:e}
\end{figure}

\paragraph{}
Each of the three has a community size of 8, 41 and 26 respectively. By looking at the size of community, we argue that \textit{Fast Greedy} tend to have larger clusters with fewer communities, and \textit{Edge-Betweeness} tend to have one very large cluster with many clusters around, and finally \textit{Infomap} is somewhat moderate between the two. We also notice that the three has some overlaps, indicated by the yellow region in Figure \ref{fig:d} and red in Figure \ref{fig:e} and \ref{fig:f}. In all three cases they are clustered into a group.

\begin{figure}[h!]
	\centering
	\includegraphics[width=.6\linewidth]{q1f.png}
	\caption{Node 1 Community Structure by \textit{Infomap algorithms}}	
	\label{fig:f}
\end{figure}