\subsection*{Section 4}
\paragraph{}
Figure \ref{fig:g} to \ref{fig:i} presents the community structure after excluding the core node. The community size are 26, 50 and 40 respectively. This indicates that the structure of the graph are more scattered by removing the core node. This can also be seen from figures directly, as the largest cluster in Figure \ref{fig:d} (yellow), Figure \ref{fig:e} (red), Figure \ref{fig:f} (red) are now partitioned into several subgroups shown in Figure \ref{fig:g} (yellow and orange), Figure \ref{fig:h} (orange and red), Figure \ref{fig:i} (red and green).


\begin{figure}[h!]
	\centering
	\includegraphics[width=.6\linewidth]{q1g.png}
	\caption{Node 1 Community Structure by \textit{fast greedy algorithms}}	
	\label{fig:g}
\end{figure}
\begin{figure}[h!]
	\centering
	\includegraphics[width=.6\linewidth]{q1h.png}
	\caption{Node 1 Community Structure by \textit{Edge-Betweenness algorithms}}	
	\label{fig:h}
\end{figure}

\begin{figure}[h!]
	\centering
	\includegraphics[width=.6\linewidth]{q1i.png}
	\caption{Node 1 Community Structure by \textit{Infomap algorithms}}	
	\label{fig:i}
\end{figure}