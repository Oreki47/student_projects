\subsection*{problem 1}
\subsubsection*{(1)}
\paragraph{}
This graph $G$ is not connected. In the case where the graph is directed, we find its strongly connected Giant Connected Component (GCC). For the entire graph it has 10501 nodes and 427486 edges, and the GCC has 10487 nodes and 427472 edges.
\subsubsection*{(2)}
\paragraph{}
Figure \ref{fig:a1} presents the "In" and "Out" degree distribution of $G$. Both distribution resembles to power-law distribution that follows fat-tail distribution, specifically with $\beta \approx 1.5$. This is presented by the red line $y = x^{-1.5}$ in both sub figure. Since $\beta \in (1,\ 3)$, we argue that graph $G$ follows heavy-tail distribution. And in this case both mean and variance of edge $k$ is unbounded.  
\vspace{-12pt}
\begin{figure}[h]
	\centering
	\begin{subfigure}{.5\textwidth}
		\centering
		\includegraphics[width=.9\linewidth]{hw3p1a3.png}
		\caption{Mean}	
	\end{subfigure}%
	\begin{subfigure}{.5\textwidth}
		\centering
		\includegraphics[width=.9\linewidth]{hw3p1a4.png}
		\caption{Standard Deviation}
	\end{subfigure}
	\caption{The" In" and "Out" degree distribution of graph $G$ with log-scale.}
	\label{fig:a1}
\end{figure}

\subsubsection*{(c)}
\paragraph{}
Table \ref{table:1} presents option 1, where number of edges remains unchanged and only direction is removed. Its community is generated by \textit{label.propagation.community} method. The modularity is $m = 0.000217338$.
\paragraph{}
Table \ref{table:2} and \ref{table:3} present option 2, where two directed edges between nodes $i$ and $j$ are merged and therefore resulting in a simple graph. Its community is generated by both \textit{label.propagation.community} and \textit{fastgreedy.community} method. Modularity of each case is $m_{lpc} =  0.000217338$ and $ m_{fgc} = 0.3284922$, with the latter showing the maximum.
\paragraph{}
Based on three cases above, we argue that \textit{Label Propagation Algorithm} are more versatile since it can be used in find community in both simple and complex graphs. However, it has a tendency to form large, loose clusters, as shown by its modularity and the node placement in community. On the other hand, \textit{Fast Greedy Algorithm} is more focused on simple graph, but performs better at clustering, resulting in a larger modularity and more evenly distributed community structure.

\begin{table}[h!]
	\centering
	\begin{tabular}{ |c|c|c|c|c|c|c|c|c|c|c|c|c| }
		\hline
		Community & 1 	  & 2 & 3 & 4 & 5 & 6 & 7 & 8 & 9 & 10 & 11 & 12  \\ 
		\hline
		Size      & 10472 & 2 & 4 & 2 & 2 & 4 & 2 & 2 & 3 & 5  & 2  & 2\\ 
		\hline
	\end{tabular}
	\caption{Community Structure generated by \textit{Label Propagation Algorithm}}
	\label{table:1}
\end{table}

\begin{table}[h!]
	\centering
	\begin{tabular}{ |c|c|c|c|c|c|c|c|c|c|c|c|c|c| }
		\hline
		Community & 1 	  & 2 & 3 & 4 & 5 & 6 & 7 & 8 & 9 & 10 & 11 & 12 & 13  \\ 
		\hline
		Size      & 10469 & 2 & 4 & 2 & 2 & 3 & 2 & 3 & 2 & 3  & 5  & 2  & 2\\ 
		\hline
	\end{tabular}
	\caption{Community Structure generated by \textit{Label Propagation Algorithm}}
	\label{table:2}
\end{table}

\begin{table}[h!]
	\centering
	\begin{tabular}{ |c|c|c|c|c|c|c|c|c|c|c|c|c|c|c|c| }
		\hline
		Community & 1 	 & 2    & 3    & 4    & 5   & 6   & 7    & 8   & 9 & 10 & 11 & 12 & 13 & 14 & 15  \\ 
		\hline
		Size      & 1842 & 1709 & 1118 & 1205 & 736 & 902 & 2333 & 642 & 2 & 2  & 2  & 2  & 2  & 2  & 2 \\ 
		\hline
	\end{tabular}
	\caption{Community Structure generated by \textit{Fast Greedy Algorithm}}
	\label{table:3}
\end{table}


\subsubsection*{(d)}
\paragraph{}
Table \ref{table:4} shows the membership of the largest community found by \textit{Fast Greedy Algorithm}. This subgraph has 2333 nodes with 22309 edges. Its modularity $m_sub=0.3624$, which is greater than the original graph, showing that this subgraph is more densely connected. Figure \ref{fig:d1} shows the Sub Community Structure of the Largest Community.

\begin{table}[h!]
	\centering
	\begin{tabular}{ |c|c|c|c|c|c|c|c|c|c|}
		\hline
		Community & 1  & 2   & 3   & 4  & 5   & 6   & 7   & 8   & 9  \\ 
		\hline
		Size      & 37 & 400 & 373 & 32 & 320 & 385 & 346 & 435 & 5  \\ 
		\hline
	\end{tabular}
	\caption{Sub Community Structure of the Largest Community}
	\label{table:4}
\end{table}

\begin{figure} [h!]
	\centering
	\includegraphics[scale=0.7]{hw3p4com.png}
	\caption{Sub Community Structure of the Largest Community}
	\label{fig:d1}
\end{figure}

\subsubsection*{(e)}
\paragraph{}
As shown in Table \ref{table:5}, there are 8 clusters with size greater than 100. Figure \ref{fig:e1} and Figure \ref{fig:e2} show the size of each cluster in each of the sub-community.
\begin{table}[h!]
	\centering
	\begin{tabular}{ |c|c|c|c|c|c|c|c|c|c|c|c|c|c|c|c| }
		\hline
		Community      & 1 	 & 2    & 3    & 4    & 5   & 6   & 7    & 8   \\ 
		\hline
		Size           & 1842 & 1709 & 1118 & 1205 & 736 & 902 & 2333 & 642 \\ 
		\hline
		Max Modularity & 0.2231 & 0.3724 & 0.5093 & 0.3966 & 0.4113 & 0.4960 & 0.3624 & 0.4797 \\ 
		\hline
	\end{tabular}
	\caption{Sub-Community structure with size $>$ 100}
	\label{table:5}
\end{table}

\begin{figure}[h!]
	\centering
	\begin{subfigure}{.5\textwidth}
		\centering
		\includegraphics[width=.9\linewidth]{hw3p5a1.png}	
	\end{subfigure}%
	\begin{subfigure}{.5\textwidth}
		\centering
		\includegraphics[width=.9\linewidth]{hw3p5a2.png}
	\end{subfigure}
	\begin{subfigure}{.5\textwidth}
		\centering
		\includegraphics[width=.9\linewidth]{hw3p5a3.png}	
	\end{subfigure}%
	\begin{subfigure}{.5\textwidth}
		\centering
		\includegraphics[width=.9\linewidth]{hw3p5a4.png}
	\end{subfigure}
	\caption{Sub-Community Structure of the First 4 Sub-Community}
	\label{fig:e1}
\end{figure}

\begin{figure}[h!]
	\centering
	\begin{subfigure}{.5\textwidth}
		\centering
		\includegraphics[width=.9\linewidth]{hw3p5a5.png}	
	\end{subfigure}%
	\begin{subfigure}{.5\textwidth}
		\centering
		\includegraphics[width=.9\linewidth]{hw3p5a6.png}
	\end{subfigure}
	\begin{subfigure}{.5\textwidth}
		\centering
		\includegraphics[width=.9\linewidth]{hw3p5a7.png}	
	\end{subfigure}%
	\begin{subfigure}{.5\textwidth}
		\centering
		\includegraphics[width=.9\linewidth]{hw3p5a8.png}
	\end{subfigure}
	\caption{Sub-Community Structure of the Last 4 Sub-Community}
	\label{fig:e2}
\end{figure}

\subsubsection*{(f)}
\paragraph{}
Table \ref{table:6} to \ref{table:8} shows the results. With \textit{Fast Greedy Algorithm} we set the threshold to 0.2 and 0.3, and with \textit{Label Propagation Algorithm} we only present the case with threshold = 0.2, since the former has more evenly distributed clusters. Notice that for the latter case, we obtained 4 nodes that all belong to Community 3 in Table \ref{table:1}
\begin{table}[h!]
	\centering
	\begin{tabular}{|r||*{8}{c|}}
		\hline
		\backslashbox{Node~}{Community~~}
			 	& 1    & 2    & 3    & 4    & 5    & 6    & 7    & 8    \\ 
		\hline
		9369    & 0.08 & 0.14 & 0.46 & 0.21 & 0.02 & 0.00 & 0.09 & 0.00 \\
		\hline
		1980    & 0.10 & 0.25 & 0.00 & 0.12 & 0.02 & 0.01 & 0.50 & 0.00 \\ 
		\hline
		1586    & 0.14 & 0.58 & 0.00 & 0.15 & 0.03 & 0.02 & 0.07 & 0.00 \\ 
		\hline
		10339   & 0.06 & 0.14 & 0.00 & 0.49 & 0.02 & 0.00 & 0.29 & 0.00 \\ 	
		\hline
	\end{tabular}
	\caption{Nodes belonging to multiple communities with threshold = 0.2, by  \textit{Fast Greedy Algorithm}}
	\label{table:6}
\end{table}

\begin{table}[h!]
	\centering
	\begin{tabular}{|r||*{8}{c|}}
		\hline
		\backslashbox{Node~}{Community~~}
				& 1    & 2    & 3    & 4    & 5    & 6    & 7    & 8    \\ 
		\hline
		5246    & 0.07 & 0.13 & 0.35 & 0.38 & 0.02 & 0.01 & 0.04 & 0.00 \\
		\hline
		2868    & 0.07 & 0.35 & 0.37 & 0.10 & 0.01 & 0.00 & 0.08 & 0.00 \\ 
		\hline
		899     & 0.08 & 0.36 & 0.36 & 0.04 & 0.02 & 0.02 & 0.01 & 0.00 \\ 
		\hline
		7750    & 0.09 & 0.35 & 0.00 & 0.06 & 0.01 & 0.01 & 0.09 & 0.39 \\ 	
		\hline
	\end{tabular}
	\caption{Nodes belonging to multiple communities with threshold = 0.3, by \textit{Fast Greedy Algorithm}}
	\label{table:7}
\end{table}

\begin{table}[]
	\centering
	\begin{tabular}{|r||*{6}{c|}}
		\hline
		\backslashbox{Node~}{Community~~}
		    	& 1    & 2    & 3    & 4    & 5    & 6     \\ 
		\hline
		4966    & 0.64 & 0.00 & 0.36 & 0.00 & 0.00 & 0.00  \\
		\hline
		4967    & 0.66 & 0.00 & 0.34 & 0.00 & 0.00 & 0.00  \\ 
		\hline
		4968    & 0.65 & 0.00 & 0.35 & 0.00 & 0.00 & 0.00  \\ 
		\hline
		4969    & 0.65 & 0.00 & 0.35 & 0.00 & 0.00 & 0.00  \\ 
		\hline
	\end{tabular}
	\caption{Nodes belonging to multiple communities with threshold = 0.2, by \textit{Label Propagation Algorithm}}
	\label{table:8}
\end{table}