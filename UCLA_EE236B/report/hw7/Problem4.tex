\subsection*{A4.10}
\subsubsection*{(a)}
\paragraph{}
Introducing Lagrange multiplier $\nu \in R^n$, and we have
\begin{align*}
&L(x, \nu) = x^TA^TAx -2b^ITAx +b^Tb +\sum_{i=1}^{n}\nu_i(x_i-1) \\
& \qquad \quad = x^T(A^TA +\textbf{diag}(\nu))x-2b^TAx +b^Tb -\nu^T\textbf{1}
\end{align*}
\paragraph{}
$L(x,\nu)$ is bounded below if $A^TA+\textbf{diag}(\nu) \succeq 0$ and $2b^TA \in \text{Range}(A^TA+\textbf{diag}(\nu))$. Take partial over $x$ and we have $x =(A^TA+\textbf{diag}(\nu))^\dagger A^Tb$ and 
\begin{align*}
g(\nu) =-b^TA(A^TA+\textbf{diag}(\nu))^\dagger A^Tb +b^Tb-\nu^T\textbf{1}
\end{align*}
\paragraph{}
With introducing $t\in R$, the dual problem is therefore equivalent to 
\begin{align*}
&maximize \qquad b^Tb-t-\nu^T\textbf{1}\\
&subject\ to \qquad \begin{bmatrix}
&A^TA+\textbf{diag}(\nu)&-A^Tb\\&-b^TA&t
\end{bmatrix} \succeq 0
\end{align*}
\paragraph{}
With $t,\ \nu$ being the variables.
\subsubsection*{(b)}
\paragraph{}
First we write it into minimization form
\begin{align*}
&minimize \qquad t+\nu^T\textbf{1}-b^Tb\\
&subject\ to \qquad \begin{bmatrix}
&A^TA+\textbf{diag}(\nu)&-A^Tb\\&-b^TA&t
\end{bmatrix} \succeq 0
\end{align*}
\paragraph{}
With introduction of Lagrange multiplier
\begin{align*}
\begin{bmatrix}
&Z&z\\&z^T&\lambda
\end{bmatrix},
\end{align*}
\paragraph{}
The Lagrangian can be written as:
\begin{align*}
&L(t,\nu,Z,z,\lambda)=t+\textbf{1}^T\nu-b^Tb-\textbf{tr}(Z(A^TA+\textbf{diag}(\nu)))+2z^TA^Tb -t\lambda\\
&\qquad \qquad \qquad = (1-\lambda)t +(\textbf{1}-\textbf{diag}(Z))^T\nu -\textbf{tr}(ZA^TA)+2b^TAz -b^Tb\\
&\qquad \qquad \qquad =\begin{cases}
&-\textbf{tr}(ZA^TA)+2b^TAz -b^Tb, \qquad \text{if}\ \textbf{diag}(Z)=\textbf{1}, \lambda=1\\
&-\infty \qquad  \qquad \qquad \qquad \qquad \qquad \ \ \text{otherwise}
\end{cases}
\end{align*}
\paragraph{}
Writing the dual function in minimization form and we have
\begin{align*}
&minimize \qquad \textbf{tr}(ZA^TA)-2b^TAz + b^Tb\\
&subject\ to \qquad \textbf{diag}(Z)=\textbf{1}\\
&\qquad \qquad \qquad \begin{bmatrix}
&Z&z\\&z^T&1
\end{bmatrix} \succeq 0
\end{align*}
\paragraph{}
To see this is a relaxation of the original problem. First we have
\begin{align*}
&||Ax-b||^2_2 = x^TA^TAx -2b^TAx+b^Tb\\
&\qquad \qquad \ \ =\textbf{tr}(x^TA^TAx) -2b^TAx +b^Tb\\
&\qquad \qquad \ \ =\textbf{tr}(A^TAxx^T) -2b^TAx +b^Tb
\end{align*}
\paragraph{}
This problem is therefore equivalent to
\begin{align*}
&minimize \qquad \textbf{tr}(ZA^TA)-2b^TAz + b^Tb\\
&subject\ to \qquad \textbf{diag}(Z)=\textbf{1}\\
&\qquad \qquad \qquad Z=zz^T
\end{align*}
\paragraph{}
This is replaced by a weaker constraint $Z\succeq zz^T$ in the SDP and therefore it is a relaxation of the original problem. When
\begin{align*}
\text{rank}( \begin{bmatrix}
&Z&z\\&z^T&1
\end{bmatrix}) = 1
\end{align*}
\paragraph{}
We have
\begin{align*}
[Z\ z] = q[z\ 1]
\end{align*}
\paragraph{}
and obviously a solution of $q = z$ and we have $Z=zz^T$.
\subsubsection*{(c)}
\begin{align*}
&\textbf{E}||A\nu-b||^2_2 =\textbf{E}[\nu^TA^TA\nu -2b^TA\nu+b^Tb]\\
&\qquad \qquad \quad \ =\textbf{E}[\nu^TA^TA\nu]- \textbf{E}[2b^TA\nu] +b^Tb\\
&\qquad \qquad \quad \ =\textbf{tr}(\textbf{E}[\nu\nu^T]A^TA)-2b^TA\textbf{E}\nu+b^Tb
\end{align*}
\paragraph{}
Therefore, the equivalence is followed by $Z = \textbf{E}[\nu\nu^T]$ and $z = \textbf{E}\nu$.
\subsubsection*{(d)}
\paragraph{}
The optimal values are as followed:
\begin{center}
	\begin{tabular}{ c | c | c | c | c | c}
		s   & $f(x_a)$ & $f(x_b)$ & $f(x_c)$ & $f(x_d)$ & $d^\star$ \\ 
		0.5 & 4.1623   & 4.1623   & 4.1623   & 4.1623   & 4.0524    \\  
		1.0 & 12.7299  & 8.3245   & 8.3245   & 8.3245   & 7.8678    \\ 
		2.0 & 30.1419  & 16.6490  & 16.6490  & 16.6490  & 15.1814   \\ 
		3.0 & 33.9339  & 25.9555  & 25.9555  & 24.9735  & 22.1139   \\ 
	\end{tabular}
\end{center}
\paragraph{}
See matlab code as follow.
\verbatiminput{bi.m}