\subsection*{Problem 7}
\paragraph{}
First we examined the actual rating of these three movies. Then we implement two methods utilizing their neighbors to predict their ratings. The results is shown in Table \ref{table:5}.

\subsubsection*{Model 1}
\paragraph{}
We start with using the average of 15 nearest neighbors of the target nodes. 
\subsubsection*{Model 2}
\paragraph{}
For model 2 we used a weighted average of 15 nearest neighbors of the target nodes, which can be represented as follows.
\begin{align*}
R(m) = \sum_{n \in N(m)} \frac{W(n)R(n)}{E[W(n)]},
\end{align*}
\paragraph{}
where $R(m)$ is the rating of the target movie $m$, $N(m)$ is the neighbors of movie $m$, $W(n)$ is the weight between movie $m$ and $n$, and $E[W(n)]$ is the expectation, or in this case, the average of the weights of movie $n$ that are neighbors of $m$.
\paragraph{}
We did not perform further analysis to this as the sample size is very small, i.e., only 3 movies. Ideally we may be able to find a function that fits really well to these 3 movies but such function may most likely be overfitting. Instead, we will expand our regression and prediction analysis in Problem 8 and 9, where we attempt to find fitting functions that apply to the entire dataset.
\begin{table}[h!]
	\centering
	\caption{Prediction of the 3 Target Movies}
	\begin{tabular}{{|c}*{4}{|c|}}
		\hline
		&Movie Name 	& Batman v Superman:     & Mission: Impossible & Minions(2015)   	\\
		&           	& Dawn of Justice(2016)  & Rogue Nation(2015)  &               		\\
		\hline
		&Actual Rating  & 7.1  	                 & 7.5                 & 6.4    			\\ 
		\hline
		&Model1 Rating  & 6.283333		         & 6.185714            & 6.742857   		\\ 
		\hline
		&Model2 Rating  & 6.279793  	         & 6.273461	           & 6.828282  			\\ 	
		\hline
	\end{tabular}
	\label{table:5}
\end{table}

