\subsection*{Problem 6}
\paragraph{}
Here, instead of setting $\tau = 1$, we set $\tau=7$, i.e., we sample data weekly on Mondays. Applying the same method, we acquired a MST with vine clusters shown in Figure \ref{fig:06mst}. Using the same metric in Problem 4, we have 
\begin{align*}
\alpha_{weekly} = 0.5482848,
\end{align*}
which is a bit smaller than daily $\alpha$. This is somewhat expected, as we increase the sample frequency, we tend to miss out more information and therefore we are less likely to discover a pattern within sectors. This results in a decrease of correlation coefficient among stocks belonging to the same sector, which in turn decrease the probability that, in the MST, the neighbors of a node and the node itself belongs to the same sector.

\begin{figure}[H]
	\centering
	\includegraphics[width=.5\linewidth]{06mst.png}
	\caption{MST by Week}	
	\label{fig:06mst} 
\end{figure}

