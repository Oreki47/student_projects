\subsubsection*{Problem 3}
\subsubsection*{(1)}
\paragraph{}
Let's name the LP with the edge setup as $\textcircled{1}$ and path setup as $\textcircled{2}$. To prove equivalence:
\paragraph{}
$\textcircled{1}$ $\Rightarrow$ $\textcircled{2}$
\paragraph{}
Given a feasible solution of $\textcircled{1}$, we can acquire a feasible set of $f_i$ through flow decomposition:
\begin{enumerate}
	\item Start with path $P_1$. For this path $P_1$ select the edge $e^{min}$ such that $e^{min}=arg \ \underset{e\in P_1}{\min} f^{\star}_e$. Set the flow of the path as $f_{e^{min}}$.
	\item Subtract $f_{e^{min}}$ from all edges that belongs to $P_1$.
	\item Remove edges from the graph that only belong to $P_1$.
	\item Repeat the process with path index +1 until all edges with flows greater than zero are assigned and removed.
\end{enumerate}
\paragraph{}
Such set $P$ of path is clearly feasible because of the flow conservation constraint in $\textcircled{1}$. To see $\textcircled{1}$ and $\textcircled{2}$ have the same optimal value, notice that each path ends up in the sink t, and therefore for each $P_i$ in $P$, there exist j such that $(j, t) \in P_i$. 
\begin{align*}
\sum f_i = \underset{j:(j, t)\in E}{\sum}f_{jt} = f_{ds}
\end{align*}
\paragraph{}
$\textcircled{2}$ $\Rightarrow$ $\textcircled{1}$
\paragraph{}
Given a set of viable path $P$. We can simply add them up to acquire a set E of edge flow. Such set E satisfies the capacity constraint because of $\underset{(i:e\in P_i)}{\sum}f_i \leq c_e$. The flow conservation constraint also holds because for $i \in V$, for all paths that go through $V_i$ we have $f_{in} = f_{out}$. And by summing up we acquire the flow conservation constraint. Same argument could be use to prove that the objective function value of the two problems are the same.
\subsubsection*{(2, 3)}
To prove the statement. We first derive the dual of $\textcircled{2}$ as follow:
\begin{align*}
& minimize \quad \ \ \  \sum C_ed_e\\
& subject \ to \qquad \underset{e\in P_i}{\sum} d_e \geq 1, \ \forall P_i\\
& \qquad \qquad \qquad  \ d_e \geq 0 \\
\end{align*}
\vspace{-42pt}
\paragraph{}
The dual of $\textcircled{2}$ is a min-cut problem with $d_e$ representing as an indicator variable whether edge e is in the cut, i.e.,
\begin{align*}
d_e = \begin{cases}
&1 \qquad e \in C^{\star}(s, t)\\
&0 \qquad \text{otherwise}.
\end{cases}
\end{align*}
\paragraph{}
Now statement in (2) is obviously true. When we have $C_e=1, \ \forall e \in E$, the objective function becomes $\sum d_e=k$ and therefore for the min-cut $C^{\star}(s, t)$ there is k edge-disjoint paths from s to t.