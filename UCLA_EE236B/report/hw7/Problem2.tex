\subsection*{T5.29}
\paragraph{}
The KKT condition goes as follow
\begin{align*}
x_1^2+x_2^2+x_3^2=1, \qquad (\nu-3)x_1 +1 = 0, \qquad (1+\nu)x_2+1 =0, \qquad (2+\nu)x_3 +1 =0.
\end{align*}
\paragraph{}
This leads to
\begin{align*}
 (\frac{1}{3-\nu})^2 +( -\frac{1}{1+\nu})^2 +(-\frac{1}{2+\nu})^2 =1 
\end{align*}
\paragraph{}
Solving this equation and we have:
\begin{align*}
\nu = 4.04, \quad \nu = 1.89, \quad \nu = 0.22, \quad \nu = -3.15
\end{align*}
\paragraph{}
Which corresponds to
\begin{align*}
&x =[-0.97\ -0.2\ -0.17]^T \quad x=[-0.9\ -0.35\ -0.26]^T \\ &x=[0.36\ -0.82\ -0.45]^T \quad \ x=[0.16\ 0.47\ 0.87]
\end{align*}
\paragraph{}
This corresponds to an objective value of
\begin{align*}
f_0 = -5.37, \quad f_0 = -1.6, \quad f_0 = -1.13, \quad f_0 = -4.65
\end{align*}
\paragraph{}
Hence, the pair that corresponds to the optimum is $\nu^\star = 4.04, x^\star =[-0.97\ -0.2\ -0.17]^T$.