\subsection*{Problem 2}
\subsubsection*{(a)}
\paragraph{}
This is the same as hw1 and is realized by $barabasi.games()$.
\subsubsection*{(b)}
\paragraph{}
Figure \ref{fig:b1} presents the results of such random walk.
\begin{figure}[h]
	\centering
	\begin{subfigure}{.5\textwidth}
		\centering
		\includegraphics[width=.9\linewidth]{hw2p2b1.png}
		\caption{Mean}
	\end{subfigure}%
	\begin{subfigure}{.5\textwidth}
		\centering
		\includegraphics[width=.9\linewidth]{hw2p2b2.png}
		\caption{Standard Deviation}
	\end{subfigure}
	\caption{Distance $\langle s(t)\rangle$, with $n = 1000,\ t = 50,\ num\_walkers = 2000$}
	\label{fig:b1}
\end{figure}

\begin{figure}[h!]
	\centering
	\begin{subfigure}{.5\textwidth}
		\centering
		\includegraphics[width=.9\linewidth]{hw2p2b3.png}
		\caption{Mean}
	\end{subfigure}%
	\begin{subfigure}{.5\textwidth}
		\centering
		\includegraphics[width=.9\linewidth]{hw2p2b4.png}
		\caption{Standard Deviation}
	\end{subfigure}
	\caption{Distance $\langle s(t)\rangle$, with $n = 100,\ t = 50,\ num\_walkers = 200$}
	\label{fig:b2}
\end{figure}

\begin{figure}[h!]
	\centering
	\begin{subfigure}{.5\textwidth}
		\centering
		\includegraphics[width=.9\linewidth]{hw2p2b5.png}
		\caption{Mean}
	\end{subfigure}%
	\begin{subfigure}{.5\textwidth}
		\centering
		\includegraphics[width=.9\linewidth]{hw2p2b6.png}
		\caption{Standard Deviation}
	\end{subfigure}
	\caption{Distance $\langle s(t)\rangle$, with $n = 10000,\ t = 50,\ num\_walkers = 2000$}
	\label{fig:b3}
\end{figure}

\subsubsection*{(c)}
\paragraph{}
This again shares no similarity to d-dimensional random walk. Essentially the walker behave similarly to the walker in question 1. Refer to (d) for a more detailed comparison
\subsubsection*{(d)}
\paragraph{}
Figure \ref{fig:b2} and \ref{fig:b3} presents the case of $n = 100$ and $n = 10000$. As a typical realization, diameter of $n = 100, 1000$ and 10000 are $d = 12, 20$ and 28 respectively. Here the diameter again provides some sort of an upper-bound and the larger the diameter, the larger the mean and standard deviation will be.
\paragraph{}
Essentially, a random graph generated by $barabasi.games()$ has fewer high-degree nodes as compared to $erdos.renyi.games()$ where degrees are more evenly distributed. This results in mean and standard deviation converging in a much longer step. Figure \ref{fig:b4} presents such an idea. With $t=2000$ and $num\_walkers =10$ (with limited computing power we selected 10 to illustrate the concept), the mean and standard deviation are, again, bounded eventually.

\begin{figure}[h!]
	\centering
	\begin{subfigure}{.5\textwidth}
		\centering
		\includegraphics[width=.9\linewidth]{hw2p2b7.png}
		\caption{Mean}
	\end{subfigure}%
	\begin{subfigure}{.5\textwidth}
		\centering
		\includegraphics[width=.9\linewidth]{hw2p2b8.png}
		\caption{Standard Deviation}
	\end{subfigure}
	\caption{Distance $\langle s(t)\rangle$, with $n = 1000,\ t = 2000,\ num\_walkers = 10$}
	\label{fig:b4}
\end{figure}

\subsubsection*{(e)}
\paragraph{}
Again, with long enough step the walker will be trapped in clusters with high degree nodes. This is shown in Figure \ref{fig:b5}


\begin{figure} [h!]
	\centering
	\includegraphics[scale=0.7]{hw2p2edeg.png}
	\caption{Degree distribution, with $n = 1000,\ t = 50,\ num\_walkers = 200$}
	\label{fig:b5}
\end{figure}