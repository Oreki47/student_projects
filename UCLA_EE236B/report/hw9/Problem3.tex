\subsection*{A8.9}
\subsubsection*{(a)}
\paragraph{}
First we rewrite the problem as 
\begin{align*}
\text{minimize}\qquad -\sum_{\bar{y_i}=1}\log \Phi(a_i^Tx -b_i)  -\sum_{\bar{y_i}=-1}\log \Phi(b_i  - a_i^Tx) 
\end{align*}
\paragraph{}
Where $\Phi(x)$ is the cumulative Gaussian distribution function, which is log-concave. Therefore this problem is convex.
\subsubsection*{(b)}
\paragraph{}
To simplify the problem, we redefine $A$ and $b$ as 
\begin{align*}
A = \textbf{diag}(y)A, \qquad b=\textbf{diag}(y)b.
\end{align*}
\paragraph{}
Thus the problem can be written as
\begin{align*}
\text{minimize}\qquad h(Ax-b),
\end{align*}
\paragraph{}
where
\begin{align*}
h(\omega) = -\sum_{i=1}^{m}\log \Phi(\omega_i).
\end{align*}
\paragraph{}
The gradient and Hessian of $f(x) = H(Ax-b)$ are
\begin{align*}
\nabla f(x) = A^T\nabla h(Ax-b),\qquad \nabla f(x)^2 = A^T\nabla^2 h(Ax-b)A 
\end{align*}
\paragraph{}
The first derivative of $h$ are
\begin{align*}
\frac{\partial h(\omega)}{\partial \omega_i} = \frac{-1/\sqrt{2\pi}}{\exp(\omega^2_i/2)\Phi(\omega_i)} 
\end{align*}
\paragraph{}
The Hessian of $h$ are
\begin{align*}
\frac{\partial^2h(\omega)}{\partial \omega_i^2} = \frac{\omega_i/\sqrt{2\pi}}{\exp(\omega^2_i/2)\Phi(\omega_i)} +(\frac{1/\sqrt{2\pi}}{\exp(\omega^2_i/2)\Phi(\omega_i)})^2
\end{align*}
\paragraph{}
This converges within 5 steps and $x =(−0.27, 9.15, 7.98, 6.70, 6.02, 5.0, 4.30, 2.68, 2.02, 0.68).$
\verbatiminput{main.m}