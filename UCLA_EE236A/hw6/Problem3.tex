\subsubsection*{Problem 3}
\subsubsection*{(1)}
\paragraph{}
First we introduce a variable $x_e$ such that:
\begin{align*}
x_e = \begin{cases}
&1 \qquad \text{e belongs to the matching}\\
&0 \qquad \text{otherwise}.
\end{cases}
\end{align*}
\paragraph{}
Therefore an ILP can be formed as follow:
\begin{align*}
& maximize \quad \ \ \ \sum_{e\in E} x_e\\
& subject \ to \qquad \sum_{e\in\delta(u)} x_e \leq 1, \ \forall u \in V\\
&\qquad \qquad \quad \quad  \ \ x_e \in \{0, 1\}, \ \forall u \in E
\end{align*}
\paragraph{}
Since the constraint matrix is TUM the ILP can be relaxed into LP as followed:
\begin{align*}
& maximize \quad \ \ \ \sum_{e\in E} x_e\\
& subject \ to \qquad \sum_{e\in\delta(u)} x_e \leq 1, \ \forall u \in V\\
&\qquad \qquad \quad \quad  \ \ x_e \geq 0, \ \forall u \in E
\end{align*}
\paragraph{}
Therefore, the dual can be written as:
\begin{align*}
& minimize \quad \ \ \ \sum_{u\in V} \lambda_u\\
& subject \ to \qquad \lambda_u + \lambda_v \geq 1, \ \forall e=(u, v) \in E\\
&\qquad \qquad \quad \quad \ \lambda_u \geq 0
\end{align*}
\paragraph{}
Again, since the constraint matrix is TUM we can write the corresponding ILP as:
\begin{align*}
& minimize \quad \ \ \ \sum_{u\in V} \lambda_u\\
& subject \ to \qquad \lambda_u + \lambda_v \geq 1, \ \forall e=(u, v) \in E\\
&\qquad \qquad \quad \quad \ \lambda_u \in \{0, 1\}
\end{align*}
\subsubsection*{(2)} 
The dual is a vertex cover problem and the objective function tries to minimize the size of the vertex cover. The first constraint says that for a certain edge $e$, at least one of the vertices is in the set. $\lambda_u$ is the indicator function such that:
\begin{align*}
\lambda_u = \begin{cases}
&1 \qquad \text{vertex u is in the cover}\\
&0 \qquad \text{otherwise}.
\end{cases}
\end{align*}
\subsubsection*{(3)} 
\paragraph{}
Since $M<|V_1|$, there is no perfect matching for this graph. Assume $|V_1|=|V_2|=n$, the size of the maximum matching is $\leq n-1$, and so is the size of the maximum cut is also $\leq n-1$. Let's call this cut $C$. We introduce $L_1:=C \cap V_1$, $L_2:= V_1 - C$, $R_1:=C \cap V_2$, $R_2:= V_2 - C$.
\paragraph{}
The capacity of cut $C$ can be written as:
\begin{align*}
capacity(S) = |L_2| + |R_1| +|(L1, R2)|
\end{align*}
\paragraph{}
Therefore we have:
\begin{align*}
n-1 = |L_2| + |R_1| +|(L1, R2)|
\end{align*}
\paragraph{}
Also since $|L_1| = n - |L_2|$,
\begin{align*}
|L_1| > |L_2| + |R_1| +|(L1, R2)|+1
\end{align*}
\paragraph{}
We also have $|N(L)| < |L_2| + |R_1| +|(L1, R2)|$ because the neighbor of $L_1$ can at most include $|(L1, R2)|$ vertices in $|R_2|$. To conclude, we have:
\begin{align*}
|L_1| > |N(L)| +1
\end{align*}
\paragraph{}
Therefore we have found such a set.
\subsubsection*{(4)}
\paragraph{}
If we have $M=|V_1|$, that means the solution returned by (1) is a perfect matching. Therefore for any subset S, there is at least one edge that match vertices in $S$ to its neighbor $N(S)$, and therefore $|S| \leq |N(S)|$ holds.